\documentclass{beamer}

% Theme
\usetheme{Madrid}
\usecolortheme{default}
\usepackage{tabularx}
%\usepackage{titling}
\definecolor{blue}{RGB}{0, 102, 204}
\setbeamercolor{structure}{fg=blue}

% Title page
\title{Statistics on further studies of students in IIT H}
\subtitle{MA4240  Applied Statistics}
\author{Shreyas Wankhede, Pradeep Mundlik, Akshitha Kola, Dhatri Reddy, Mouktika Cherukupalli, Avinash Malothu, Sumeeth Kumar, Saathvika Marri}
\institute{IIT Hyderabad}
\date{\today}

% Begin document
\begin{document}

% Title slide
\begin{frame}
  \titlepage
\end{frame}

% Outline slide
\begin{frame}
  \frametitle{Outline}
  \tableofcontents
\end{frame}

% Content slides
\section{Introduction}
\begin{frame}
  \frametitle{Introduction}
  This is the introduction slide.
\end{frame}

\section{Data Visualization}
\begin{frame}
  \frametitle{Data visualization}
  
\end{frame}

\section{Hypothesis Testing}
\begin{frame}
  \frametitle{Hypothesis Testing}
  \begin{block}{\textbf{Case 4:} Hypothesized proportion testing if there is enough evidence that the proportions of people opting for masters,MBA,Phd are not all equal}
    Sample data :
    \begin{tabular}{|c|c|c|c|}
        \hline
        Masters & MBA & PhD & Total\\
        \hline
        50 & 24 & 13 & 87 \\
        \hline
    \end{tabular}\\

    Let $P_{Ms}$, $P_{MBA}$, $P_{PhD}$ denote proportions of students willing to pursue Masters, MBA, PhD for higher studies \\

    $H_{0}$ :  $P_{Ms} = P_{MBA} = P_{PhD} = \dfrac{1}{3}$ \space \space \space \space $H_{a}$ : atleast one $P \neq \dfrac{1}{3}$\space \space \space \space $\alpha =0.05$\\

    Also,
    \begin{equation}
        E = \dfrac{1}{3} x 87 = 29
    \end{equation}
    and,
    \begin{equation}
        \chi^{2} = \sum\dfrac{(O - E)^{2}}{E}
    \end{equation}  
    \end{block}
\end{frame}

%\section{Hypothesis Testing}
\begin{frame}
  \frametitle{Hypothesis Testing}
  \begin{block}{\textbf{Case 4:} Case 4 continued}
   
    \begin{equation}
        \begin{aligned}
            \chi^{2} &= \dfrac{(50 - 29)^{2}}{29} + \dfrac{(24 -29)^{2}}{29} + \dfrac{(13 -29)^{2}}{29}\\
            &= 15.2 + 0.862 + 8.827\\
            &= 24.889 
        \end{aligned}
    \end{equation}
    At $df = 3 - 1 = 2 $ , p value $= 0.0001  $\\
    p value $< \alpha = 0.05$\\
    Hence there is enough evidence that population proportions are not all equal.


    \end{block}
\end{frame}

% End document
\end{document}
